\documentclass[12pt]{uoitthesis}
\usepackage{setspace}
\usepackage{texshade}
\usepackage[chapter]{algorithm}
\usepackage{algpseudocode}
\usepackage{notoccite}
\usepackage{cite}
\usepackage{array}
\usepackage{keyval}
\usepackage{chngpage}
\usepackage{float}
\usepackage{multirow}
\usepackage{graphicx}
\usepackage{epstopdf}
\usepackage{fancyhdr}
\usepackage{footmisc}
\usepackage{listings}
\usepackage{fancyvrb}
\usepackage{subfigure}
\usepackage{amsmath}
\usepackage{amsthm}
\usepackage{amssymb}
\usepackage{nomencl}
\usepackage{setspace}
\usepackage{pdfpages}
\usepackage{rotating}
\usepackage{bigstrut}
\usepackage{multirow}
\usepackage{booktabs}
\usepackage{pdflscape} % Needed to change some pages to landscape
\usepackage[T1]{fontenc}
\usepackage{textcomp} % Needed to get some symbols
\usepackage{eqnarray} % Needed to organize equations
\usepackage{tabularx} % See if that fixes tables
\usepackage{siunitx}
\sisetup{group-digits = true}
\sisetup{group-minimum-digits = 4}
\sisetup{group-separator = {,}}

\usepackage{acronym} % Loads chemical equation module... use \ce{UO2} in text
\usepackage[version=3]{mhchem} % Handle abbreviations
\usepackage{url} % Needed for url field
\usepackage[hidelinks]{hyperref}
\usepackage{sidecap} % Needed for SCfigures - side captions
\usepackage{wrapfig} % Needed for wrapped text figures
\usepackage[usenames,dvipsnames,svgnames,table]{xcolor} % Needed for coloured text
\usepackage{enumitem} % for lists with no spacing
\usepackage{longtable} % Needed for tables spanning more than one page
\usepackage{ltablex}
\usepackage{gensymb}
%\usepackage{colortbl} % coloured cells in tables
% LongTable PDF Package Documentation: ftp://ftp.tex.ac.uk/tex-archive/macros/latex/required/tools/longtable.pdf 
% Tutorial with examples can be found here: http://users.sdsc.edu/~ssmallen/latex/longtable.html
\usepackage{courier} % Used for inline code font


\setlength{\LTcapwidth}{6in}


   \makeatletter
    \def\thebibliography#1{\chapter*{References\@mkboth
      {REFERENCES}{REFERENCES}}\list
      {[\arabic{enumi}]}{\settowidth\labelwidth{[#1]}\leftmargin\labelwidth
	\advance\leftmargin\labelsep
	\usecounter{enumi}}
	\def\newblock{\hskip .11em plus .33em minus .07em}
	\sloppy\clubpenalty4000\widowpenalty4000
	\sfcode`\.=1000\relax}
	\def\lst@MSkipToFirst{%
    \global\advance\lst@lineno\@ne
    \ifnum \lst@lineno=\lst@firstline
        \def\lst@next{\lst@LeaveMode \global\lst@newlines\z@
        \lst@OnceAtEOL \global\let\lst@OnceAtEOL\@empty
        \ifnum \c@lstnumber>0
            \vspace{2 mm}
        \fi
        \lst@InitLstNumber % Added to work with modified \lsthk@PreInit.
        \lsthk@InitVarsBOL
        \c@lstnumber=\numexpr-1+\lst@lineno % this enforces the displayed line numbers to always be the input line numbers
        \lst@BOLGobble}%
        \expandafter\lst@next
    \fi}
    \makeatother
    
    

% Fill in your information
\title{Development of a Proof of Concept Prototype for a Mobile Autonomous Shotcrete and Scanning System}
\author{Michael Wrock, B.Eng., M.A.Sc.}
\predegree{Bachelor of Engineering (Hons), University of Ontario Institute of Technology, 2008}
\predegrees{Master of Applied Science, University of Ontario Institute of Technology, 2011}
\degreename{Doctor of Philosophy}
\gau{The Faculty of Engineering and Applied Science}
\program{Mechanical Engineering}
\supervisor{Scott Nokleby, Faculty of Engineering and Applied Science}
\examboard{Remon Pop-Iliev, Faculty of Engineering and Applied Science}
\examboardx{Ed Waller, Faculty of Energy Systems and Nuclear Science}
\uniboard{Jaho Seo, Department Of Automotive, Mechanical And Manufacturing Engineering}
\externalexam{John Hayes, Department of Mechanical and Aerospace Engineering}

% Gets the month as a word
\newcommand{\MONTH}{%
  \ifcase\the\month
  \or January% 1
  \or February% 2
  \or March% 3
  \or April% 4
  \or May% 5
  \or June% 6
  \or July% 7
  \or August% 8
  \or September% 9
  \or October% 10
  \or November% 11
  \or December% 12
  \fi}

% Automatically Set the month and year but you can override it
\date{\MONTH ~ \the\year}
\copyrightyear{\the\year}
\setlength\parindent{0pt}
\newtheorem{theorem}{Theorem}[section]
\newtheorem{definition}{Definition}[section]
\newtheorem{lemma}{Lemma}[section]
\newtheorem{notation}{Notation}[section]


%% Helps generate the nomenclature
\newcommand*{\glossaryname}{Glossary}
\usepackage[nonumberlist,toc]{glossaries}
\newacronym{abbr}{ABBR}{Abbreviation}

\makeglossaries

%% make dots on all table of contents
%\usepackage{tocloft}
%\renewcommand{\cftpartleader}{\cftdontfill{\cftdotsep}} % for parts
%\renewcommand{\cftchapleader}{\cftdotfill{\cftdotsep}} % for chapters

%  \newcommand{name}[num][default]{definition}

\makeatletter
% start with some helper code
% This is the vertical rule that is inserted    
     
\newcommand*{\algrule}[1][\algorithmicindent]{%
  \makebox[#1][l]{%
    \hspace*{.2em}% <------------- This is where the rule starts from
    \vrule height .75\baselineskip depth .25\baselineskip
  }
}

\newcount\ALG@printindent@tempcnta
\def\ALG@printindent{%
    \ifnum \theALG@nested>0% is there anything to print
    \ifx\ALG@text\ALG@x@notext% is this an end group without any text?
    % do nothing
    \else
    \unskip
    % draw a rule for each indent level
    \ALG@printindent@tempcnta=1
    \loop
    \algrule[\csname ALG@ind@\the\ALG@printindent@tempcnta\endcsname]%
    \advance \ALG@printindent@tempcnta 1
    \ifnum \ALG@printindent@tempcnta<\numexpr\theALG@nested+1\relax
    \repeat
    \fi
    \fi
}
% the following line injects our new indent handling code in place of the default spacing
\patchcmd{\ALG@doentity}{\noindent\hskip\ALG@tlm}{\ALG@printindent}{}{\errmessage{failed to patch}}
\patchcmd{\ALG@doentity}{\item[]\nointerlineskip}{}{}{} % no spurious vertical space
% end vertical rule patch for algorithmicx
\makeatother

 % \newcommand{\node}[1]{\texttt{\detokenize{#1}}}
  \newcommand{\node}[1]{
   \unskip\path{#1}\unskip
  }
  \newcommand{\var}[1]{
   \unskip\textit{\detokenize{#1}}\unskip
  }
  \newcommand{\cpnode}{\node{control\_panel}}
  
  \lstdefinestyle{mystyle}{
  basicstyle=\footnotesize,        % the size of the fonts that are used for the code
  breakatwhitespace=true,         % sets if automatic breaks should only happen at whitespace
  breaklines=true,                 % sets automatic line breaking
  captionpos=t,                    % sets the caption-position to bottom
  commentstyle=\color{purple},    % comment style
  frame=single,	                   % adds a frame around the code
  keepspaces=true,                 % keeps spaces in text, useful for keeping indentation of code (possibly needs columns=flexible)
  keywordstyle=\color{blue},       % keyword style  % the language of the code
  numbers=left,                    % where to put the line-numbers; possible values are (none, left, right)
  numbersep=5pt,                   % how far the line-numbers are from the code
  numberstyle=\tiny\color{black!50}, % the style that is used for the line-numbers
  rulecolor=\color{black},         % if not set, the frame-color may be changed on line-breaks within not-black text (e.g. comments (green here))
  showspaces=false,                % show spaces everywhere adding particular underscores; it overrides 'showstringspaces'
  showstringspaces=false,          % underline spaces within strings only
  showtabs=false,                  % show tabs within strings adding particular underscores
  stepnumber=1,                    % the step between two line-numbers. If it's 1, each line will be numbered
  stringstyle=\color{JungleGreen},     % string literal style
  tabsize=2,	                         % show the filename of files included with \lstinputlisting; also try caption instead of title
  upquote=true,                 % the language of the code
  morekeywords={*,Header,Eigen,Vector3f,uint32,float64,int32,float32,uint16,bool,find_package,generate_dynamic_reconfigure_options,
  add_message_files,add_service_files,add_executable,target_link_libraries,set_target_properties,
  add_dependencies,find_package,string},  
  backgroundcolor=\color{black!5}
  }
  \lstdefinestyle{C++style}{
  language = C++,
  style = mystyle
  }
  \lstdefinestyle{xmlstyle}{
  language = \acrshort{xml},
  style = mystyle
  }
  \lstdefinestyle{pythonstyle}{
  language = python,
  style = mystyle
  }
  \newcommand{\includecode}[3][C++style]{\lstinputlisting[
  style=#1,
  caption=#3,
  ]{#2}}
  \newcommand{\includecodelabel}[4][C++style]{\lstinputlisting[
  style=#1,
  caption=#3,
  label=#4,
  ]{#2}}
  \newcommand{\includecodelines}[4][C++style]{\lstinputlisting[
  style=#1,
  caption=#3 (lines {#4}),
  linerange={#4},
  ]{#2}}
  
\newcommand{\tabitem}{\textbullet~~}
  %\lstset{rangeprefix=@@@,rangesuffix=@@@,includerangemarker=false}
  \renewcommand{\lstlistingname}{\acrshort{ros} File}% Listing -> Algorithm
\renewcommand{\lstlistlistingname}{List of \lstlistingname s}% List of Listings -> List of Algorithms
\newcolumntype{L}[1]{>{\raggedright\arraybackslash}m{#1}}
  \newenvironment{myitemize}
{ \begin{itemize}
    \setlength{\itemsep}{0pt}
    \setlength{\parskip}{0pt}
    \setlength{\parsep}{0pt}     }
{ \end{itemize}                  }


\begin{document}

 % \uoittitlepage

  \setcounter{secnumdepth}{3} \setcounter{tocdepth}{3}
  \pagenumbering{roman} \setcounter{page}{1}
  \doublespacing


  %%-------------Change single space to double space--
  \newpage
  \doublespacing
  \pagenumbering{arabic} \setcounter{page}{1}

  %%-------------Include your chapters----------------
  %% You can rename the files.
  \chapter{Operation Instructions}
\label{chap:manual}
This chapter is intended to serve as a guide to give the reader better understanding of how the MASS performs the following tasks:
\begin{myitemize}
\item Setting the Home Coordinate Frame
\begin{myitemize}
\item Navigating The Robot to Home Position
\item Performing a Localization Scan
\end{myitemize}
\item Recording the Fiducial Marker
\begin{myitemize}
\item Selecting marker Keypoints
\item Training Additional Markers
\end{myitemize}
\item Shotcrete Application
\begin{myitemize}
\item Applying Shotcere to a Selected Area
\item Autonomous Shotcrete Application
\end{myitemize}
\item Shotcrete Thickness Estimation
\begin{myitemize}
\item Automatic Thickness Estimation
\item Thickness Estimation From File
\item Loading a Point Cloud From Disk
\item Estimating Shotcrete Thickness Between Two Selected Areas
\end{myitemize}
\end{myitemize}
\newpage
\begin{tabularx}{\textwidth}{p{0.3\textwidth} p{0.7\textwidth} }
    \multicolumn{2}{c}{\textbf{Setting the home coordinate frame}}\\ \toprule
    \multicolumn{2}{l}{\textbf{Navigating robot to home position}}\\ \midrule
     \multicolumn{2}{l}{
     \begin{minipage}{\textwidth} 	
\scriptsize
     \textbf{Manual Navigation} The MASS can be driven using a standard USB joystick connected to any computer on the same network that hes been configured and running \acrshort{ros}.
     \end{minipage}
     }\\
      
      \\
\begin{minipage}{.3\textwidth} 	
\scriptsize
\raggedright
        \textbf{Autonomous Navigation} The MASS can autonomously navigate to a goal specified using the 2D Nav Goal tool located at the top of the RViz window. The MASS performs \acrshort{slam} as it is navigating and is capable of avoiding obstacles and planning paths around them. To set a goal click the RViz window where the robot should to navigate to, and drag the cursor in the direction the robot should face. A green arrow is shown representing the final position and orientation the navigation system will attempt to achieve. Though not required, it is recommended the operator perform a non-homing scan (by clicking the ``Scan'' button) so RViz can display a rendering of the environment.
      \end{minipage}%
      &
        \begin{minipage}{.7\textwidth}
        \vspace{1pt}
      \begin{center}
            \includegraphics[width=\linewidth]{Pics/Manual/afar_goal.png}
      \captionof{figure}{Setting a Navigation Goal}
      \label{fig:goal}
		\end{center}
    \end{minipage}
\end{tabularx}

\begin{tabularx}{\textwidth}{m{0.3\textwidth} m{0.7\textwidth} }
    \multicolumn{2}{l}{\textbf{Performing a Homing Scan}}\\ \midrule
\begin{minipage}{.3\textwidth} 	
\scriptsize
\raggedright
       Click the Set Home button on the Control Panel. All future scans will be localized to this coordinate frame.
      \end{minipage}%
      &
        \begin{minipage}{.7\textwidth}
        \vspace{1pt}
      \begin{center}
            \includegraphics[width=.5\linewidth]{Pics/Manual/sethome.png}
      \captionsetup[figure]{font=scriptsize}
      \captionof{figure}{Control Panel, Operation Tab. Set Home Button Highlighted}
		\end{center}
    \end{minipage}
\end{tabularx}
\newpage
\begin{tabularx}{\textwidth}{p{0.3\textwidth} p{0.7\textwidth} }
    \multicolumn{2}{c}{\textbf{Recording the Fiducial Marker}}\\ \toprule
    \multicolumn{2}{l}{\textbf{Selecting Marker Keypoints}}\\ \midrule
\begin{minipage}{.3\textwidth} 	
\scriptsize
\raggedright
       Using the Selection Tool, click and drag to form a box surrounding the marker keypoint.
      \end{minipage}%
      &
        \begin{minipage}{.7\textwidth}
        \vspace{1pt}
      \begin{center}
            \includegraphics[width=\linewidth]{Pics/Manual/marker1_selecting.png}
      \captionof{figure}{Selecting a Marker Keypoint}
      \label{fig:afar}
		\end{center}
    \end{minipage}\\
    \multicolumn{2}{c}{\begin{minipage}{\textwidth}
        \vspace{1pt}
      \begin{center}
            \includegraphics[width=\linewidth]{Pics/Manual/marker_2view.png}
      \captionof{figure}{Marker Keypoints Are Easily Visible When Colours Are Set to Indicate Intensity}
		\end{center}
    \end{minipage}}
\end{tabularx}

\begin{tabularx}{\textwidth}{m{0.3\textwidth} m{0.7\textwidth} }
 \multicolumn{2}{l}{\textbf{Selecting Marker Keypoints (Continued)}}\\ \midrule
 \begin{minipage}{.3\textwidth} 	
\scriptsize
\raggedright
       The selected points are highlighted in blue. Points with intensity below ``intensity\_min'' (which can be changed using \node{rqt_reconfigure}) are automatically removed.
      \end{minipage}%
      &
        \begin{minipage}{.7\textwidth}
        \vspace{1pt}
      \begin{center}
            \includegraphics[width=\linewidth]{Pics/Manual/marker1_selected.png}
      \captionof{figure}{Selected Points Highlighted in Blue}
		\end{center}
        \vspace{1pt}
      \begin{center}
            \includegraphics[width=\linewidth]{Pics/Manual/marker1_selected_zoom.png}
      \captionof{figure}{A Closer Look at Selected Points}
		\end{center}
    \end{minipage}%
\end{tabularx}
\begin{tabularx}{\textwidth}{m{0.3\textwidth} m{0.7\textwidth} }
 \multicolumn{2}{l}{\textbf{Selecting Marker Keypoints (Continued)}}\\ \midrule
 \begin{minipage}{.3\textwidth} 	
\scriptsize
\raggedright
      Click the ``Cluster Pt. 1'' button.
      \end{minipage}%
      &
        \begin{minipage}{.7\textwidth}
        \vspace{1pt}
      \begin{center}
            \includegraphics[width=.5\linewidth]{Pics/Manual/cluster1.png}
      \captionof{figure}{Teaching the First Marker Keypoint}
		\end{center}
    \end{minipage}\\
     \begin{minipage}{.3\textwidth} 	
\scriptsize
\raggedright
      Repeat the process for the second and third marker keypoints.
      \end{minipage}%
      &
        \begin{minipage}{.7\textwidth}
        \vspace{1pt}
      \begin{center}
            \includegraphics[width=.5\linewidth]{Pics/Manual/cluster2.png}
      \captionof{figure}{Teaching the Second Marker Keypoint}
		\end{center}
        \vspace{1pt}
      \begin{center}
            \includegraphics[width=.5\linewidth]{Pics/Manual/cluster3.png}
      \captionof{figure}{Teaching the Third Marker Keypoint}
		\end{center}
    \end{minipage}%
\end{tabularx}
\begin{tabularx}{\textwidth}{m{0.3\textwidth} m{0.7\textwidth} }
 \multicolumn{2}{l}{\textbf{Selecting Marker Keypoints (Continued)}}\\ \midrule
         \begin{minipage}{.3\textwidth} 	
\scriptsize
\raggedright
      Once Complete, the text displays ``Marker Recorded''. The ``Auto Localize'' feature is automatically enabled, as indicated by the checkbox.
      \end{minipage}%
      &
        \begin{minipage}{.7\textwidth}
        \vspace{1pt}
      \begin{center}
            \includegraphics[width=.5\linewidth]{Pics/Manual/operation_recorded.png}
      \captionof{figure}{Fiducial Marker Completed}
		\end{center}
    \end{minipage}
\end{tabularx}
\begin{tabularx}{\textwidth}{p{0.5\textwidth} p{0.5\textwidth} }
    \multicolumn{2}{l}{\textbf{Training Additional Markers}}\\ \midrule
\begin{minipage}{.5\textwidth} 	
\scriptsize
\raggedright
       Multiple fiducial markers can be used, the localization algorithm will select the best keypoints to use for localization. All fiducial markers are stored in a file at the same location as the point clouds are automatically saved. The point cloud and fiducial marker save location is indicated by the ``Point cloud save location'' text box in the operation tab of the Control Panel. Additional markers can be added using the same procedure as the first marker. New markers must be added using a localized point cloud if the operator requires all point clouds to be localized to the same coordinate frame. To localize a point cloud and teach a new marker, the robot must take a scan from a position that includes both the old marker keypoints as well as the new ones to be taught. Figure \ref{fig:afar} shows a point cloud scan taken from a location in which two individual fiducial markers can be taught.
      \end{minipage}%
      &
        \begin{minipage}{.5\textwidth}
        \vspace{1pt}
      \begin{center}
            \includegraphics[width=.8\linewidth]{Pics/Manual/operation_save.png}
      \captionof{figure}{Point Cloud Save Location is the Same as the Marker Save Location}
		\end{center}
    \end{minipage}
\end{tabularx}

\begin{tabularx}{\textwidth}{p{0.3\textwidth} p{0.7\textwidth} }
    \multicolumn{2}{c}{\textbf{Shotcrete Application}}\\ \toprule
    \multicolumn{2}{l}{\textbf{Applying Shotcrete to a Selected Area}}\\ \midrule
\begin{minipage}{.3\textwidth} 	
\scriptsize
\raggedright
       To autonomously apply shotcrete to a selected area, use the Selection Tool located at the top of the RViz window and drag a box to encompass the desired area to apply shotcrete. More complex selection shapes can be achieved by holding the Ctrl button while selecting additional areas or points.
      \end{minipage}%
      &
        \begin{minipage}{.7\textwidth}
        \vspace{1pt}
      \begin{center}
            \includegraphics[width=\linewidth]{Pics/Manual/shotcrete_selecting.png}
      \captionof{figure}{Manual Selection of Shotcrete Area}
		\end{center}
    \end{minipage}\\
		\begin{minipage}{.3\textwidth} 	
\scriptsize
\raggedright
       Once the desired area has been selected, press the Generate Trajectory button to generate a trajectory for the manipulator to follow.\\
       \vspace{2pt}
       \includegraphics[width=\linewidth]{Pics/Manual/operation_gen.png}
      \captionsetup[figure]{font=scriptsize}
      \captionof{figure}{Generate Trajectory Button}
      \end{minipage}%
      &
        \begin{minipage}{.7\textwidth}
        \vspace{1pt}
      \begin{center}
            \includegraphics[width=\linewidth]{Pics/Manual/shotcrete_trajectory.png}
      \captionof{figure}{Trajectory Generated from Shotcrete Selection}
		\end{center}
    \end{minipage}
\end{tabularx}

\begin{tabularx}{\textwidth}{p{0.3\textwidth} p{0.7\textwidth} }
    \multicolumn{2}{l}{\textbf{Applying Shotcrete to a Selected Area (Continued)}}\\ \midrule
\begin{minipage}{.3\textwidth} 	
\scriptsize
\raggedright
       Inspect the trajectory to ensure a satisfactory result. Via points outside the robot's workspace are shown in red, while via points within the workspace are shown in green. The path along the surface the robot will follow is shown in red, the path the end effector will follow is shown in green, and the surface normals are shown in blue.
      \end{minipage}%
      &
        \begin{minipage}{.7\textwidth}
        \vspace{1pt}
      \begin{center}
            \includegraphics[width=\linewidth]{Pics/Manual/goodbadpts_robot.png}
      \captionof{figure}{Robot Workspace (White) with Via Points Highlighted in Red (Outside Workspace) or Green (Within Workspace)}
		\end{center}
    \end{minipage}\\
		\begin{minipage}{.3\textwidth} 	
\scriptsize
\raggedright
       If the trajectory is satisfactory the operator can press the Apply Shotcrete button. Once pressed, the via points outside the robot's workspace are moved and the robot will navigate to the wall if necessary. As the algorithm executes the trajectory, the blue lines connecting the surface path to the via points are turned white as each via point is completed.
      \end{minipage}%
      &
        \begin{minipage}{.7\textwidth}
        \vspace{1pt}
      \begin{center}
            \includegraphics[width=.5\linewidth]{Pics/Manual/operation_apply.png}
      \captionof{figure}{Apply Shotcrete Button}
		\end{center}
    \end{minipage}
\end{tabularx}
%
\newpage
\begin{tabularx}{\textwidth}{p{\textwidth}}
    \textbf{Autonomous Shotcrete Application}\\ \midrule
\begin{minipage}{.7\textwidth} 	
\scriptsize
\raggedright
       When the robot autonomously applies shotcrete it acts similar to a wall following robot; it will follow the mine surface beside it until the operator commands it to stop (Using the STOP button on the Control Panel).
\end{minipage}%
\begin{minipage}{.3\textwidth}
        \vspace{1pt}
      \begin{center}
            \includegraphics[width=.8\linewidth]{Pics/Manual/operation_stop.png}
      %\captionsetup[figure]{font=scriptsize}
      %\captionof{figure}{STOP Button Can be Used During Regular Operation or During Emergency Operator Intervention}
		\end{center}
		\end{minipage}\\
      \begin{center}
            \includegraphics[width=\linewidth]{Pics/Manual/3pose_frame.png}
      \captionof{figure}{Multiple Trajectories Generated as Robot Autonomously Follows Mine Applying Shotcrete}
		\end{center}\\
    \begin{minipage}{.3\textwidth} 	
\scriptsize
\raggedright
       The operator should navigate the robot to a location where the automatic shotcreting can begin, using the Set Goal tool as shown in Figure \ref{fig:goal}. Once there, the operator should ensure the automatic shotcrete selection area, as well as the automatic occlusion area is suitable. The area of the mine scan that is kept for trajectory generation is shown in blue and can be changed using the \node{rqt_reconfigure} panel.
      \end{minipage}%
        \begin{minipage}{.7\textwidth}
        \vspace{1pt}
      \begin{center}
            \includegraphics[width=\linewidth]{Pics/Manual/autokeep.png}
      \captionof{figure}{Points to be Kept for Automatic Shotcrete Application}
		\end{center}
    \end{minipage}
\end{tabularx}

\begin{tabularx}{\textwidth}{p{0.3\textwidth} p{0.7\textwidth} }
    \multicolumn{2}{l}{\textbf{Autonomous Shotcrete Application (Continued)}}\\ \midrule
    \begin{minipage}{.3\textwidth} 	
\scriptsize
\raggedright
     To ensure the robot does not attempt to apply shotcrete to itself it is important to ensure the occlusion zone is set appropriately. Using \node{rqt_reconfigure}, the area to be cropped in which the robot's components may be detected by the \acrshort{lidar} scan is set. The volume enclosed by orange lines will be automatically removed before shotcrete trajectory generation.
      \end{minipage}%
      &
        \begin{minipage}{.7\textwidth}
        \vspace{1pt}
      \begin{center}
            \includegraphics[width=\linewidth]{Pics/Manual/auto_occlude.png}
      \captionof{figure}{Points to be Removed Before Trajectory Generation}
		\end{center}
    \end{minipage}\\
    \begin{minipage}{.3\textwidth} 	
\scriptsize
\raggedright
       If no area of the mine has been selected for manual shotcrete application, the operator can click the Apply Shotcrete button to begin the automatic shotcrete application.
      \end{minipage}%
      &
        \begin{minipage}{.7\textwidth}
        \vspace{1pt}
      \begin{center}
            \includegraphics[width=.5\linewidth]{Pics/Manual/operation_apply.png}
      \captionof{figure}{Clicking Apply Shotcrete Without a Selection Begins the Autonomous Shotcreting Process}
		\end{center}
    \end{minipage}
\end{tabularx}


\begin{tabularx}{\textwidth}{p{0.3\textwidth} p{0.7\textwidth} }
    \multicolumn{2}{c}{\textbf{Shotcrete Thickness Estimation}}\\ \toprule
    \multicolumn{2}{l}{\textbf{Automatic Thickness Estimation}}\\ \midrule
\begin{minipage}{.3\textwidth} 	
\scriptsize
\raggedright
       During normal operation, the shotcrete thickness is automatically estimated after shotcrete has been applied.
      \end{minipage}%
      &
        \begin{minipage}{.7\textwidth}
        \vspace{1pt}
      \begin{center}
            \includegraphics[width=\linewidth]{Pics/Manual/thickness_result.png}
      \captionof{figure}{Automatic Thickness Estimate}\label{fig:thickeg}
		\end{center}
    \end{minipage}
\end{tabularx}

\begin{tabularx}{\textwidth}{p{0.3\textwidth} p{0.7\textwidth} }
    \multicolumn{2}{l}{\textbf{Thickness Estimation From File}}\\ \midrule
\begin{minipage}{.3\textwidth} 	
\scriptsize
\raggedright
       To apply the shotcrete thickness estimation algorithm to point clouds saved to disk, use the Thickness Estimation tab of the Control Panel to indicate the location of the initial and final scans, then click the Estimate Thickness button. The resulting estimate can be saved by clicking the save button.
      \end{minipage}%
      &
        \begin{minipage}{.7\textwidth}
        \vspace{1pt}
      \begin{center}
            \includegraphics[width=.5\linewidth]{Pics/Manual/thickness_estimate.png}
      \captionof{figure}{Estimating Shotcrete Thickness From two Saved Scans}
		\end{center}
    \end{minipage}\\
    \multicolumn{2}{l}{\textbf{Loading a Point Cloud From Disk}}\\ \midrule
    \begin{minipage}{.3\textwidth} 	
\scriptsize
\raggedright
       To load a point cloud, scan, or thickness estimate from disk click the Load button on the Operation tab of the Control Panel.
      \end{minipage}%
      &
        \begin{minipage}{.7\textwidth}
        \vspace{1pt}
      \begin{center}
            \includegraphics[width=.5\linewidth]{Pics/Manual/operation_load.png}
      \captionof{figure}{Point Cloud Files can be Loaded From Disk}
      \label{fig:load}
		\end{center}
    \end{minipage}
\end{tabularx}

\begin{tabularx}{\textwidth}{p{0.3\textwidth} p{0.7\textwidth} }
    \multicolumn{2}{l}{\textbf{Estimating Shotcrete Thickness Between Two Selected Areas}}\\ \midrule
\begin{minipage}{.3\textwidth} 	
\scriptsize
\raggedright
       To estimate shotcrete thickness for a specific area, the first point cloud should be loaded from disk (see Figure \ref{fig:load}). Using the Selection Tool, select an area slightly larger than the desired measurement area and click Set Initial Cloud. Load the final point cloud, and select the area in which to estimate thickness. Click the Set Final Cloud button. Once the initial and final point cloud have been set the user can click Calculate Thickness to generate a thickness estimate like the one shown in Figure \ref{fig:thickeg}.
      \end{minipage}%
      &
        \begin{minipage}{.7\textwidth}
        \vspace{1pt}
      \begin{center}
            \includegraphics[width=.5\linewidth]{Pics/Manual/thickness_selection.png}
      \captionof{figure}{Calculating Shotcrete Thickness for a Specific Selection}
		\end{center}
    \end{minipage}\\
    \multicolumn{2}{c}{
    \begin{minipage}{.5\textwidth}
        \vspace{1pt}
      \begin{center}
            \includegraphics[width=\linewidth]{Pics/Manual/pre_selecting.png}
      \begin{minipage}{.8\linewidth}
      \captionof{figure}{Selecting the Initial Point Cloud}
      \end{minipage}
		\end{center}
    \end{minipage}%
    \begin{minipage}{.5\textwidth}
        \vspace{1pt}
      \begin{center}
            \includegraphics[width=\linewidth]{Pics/Manual/post_selecting.png}
      \begin{minipage}{.8\linewidth}
      \captionof{figure}{Selecting the Final Point Cloud}
      \end{minipage}
		\end{center}
    \end{minipage}
    }\\
        \multicolumn{2}{c}{
    \begin{minipage}{.5\textwidth}
        \vspace{1pt}
      \begin{center}
            \includegraphics[width=\linewidth]{Pics/Manual/pre_selected.png}
      \begin{minipage}{.8\linewidth}
      \captionof{figure}{Selecting the Initial Point Cloud (V2)}
      \end{minipage}
		\end{center}
    \end{minipage}%
    \begin{minipage}{.5\textwidth}
        \vspace{1pt}
      \begin{center}
            \includegraphics[width=\linewidth]{Pics/Manual/post_selected.png}
      \begin{minipage}{.8\linewidth}
      \captionof{figure}{Selecting the Final Point Cloud (V2)}
      \end{minipage}
		\end{center}
    \end{minipage}
    }
\end{tabularx}\\




  
 
\end{document}
