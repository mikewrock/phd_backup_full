\chapter{Testing and Results}
\label{chap:testing}

\section{Mock Mine}
\label{sec:mine}
\begin{figure}
    \centering
    \includegraphics[width=\textwidth]{Pics/mockmine.jpg}
    \caption{\acrshort{mars} Lab Mock Mine}
    \label{fig:mockmine}
\end{figure}
Testing of the MASS occurred in the Mechatronic and Robotic Systems (\acrshort{mars}) Laboratory mock mine at \acrshort{uoit}. Figure \ref{fig:mockmine} shows a picture of the mock mine used for testing. The mine is 6 m long, 3 m wide, 1.5 m tall, and the roof overhang is 1 m. Since Cameco's MacArthur River mine corridors are approximately 6 m tall and 5-8 m wide, the mock mine is considered a quarter scale representation. It is constructed from wood framing with foam surfaces and features. Figure \ref{fig:surfacefeature} shows an image of some of the surface features installed on the mine. The uneven surfaces and features are similar to what can be found in the MacArthur River mine. Due to practicality, the surface roughness is much smoother than the actual mine. The main difference with the lack of surface roughness is the normal estimation will be more accurate over smaller regions, but using a larger radius to calculate the surface normals will yield similar results. Through testing it was determined the measurement error of the \acrshort{lidar} produced results similar to what would be obtained under ideal conditions with realistic surface roughness. Currently the mock mine floor is plywood material, but it has been designed to support a gravel surface. The gravel adds an additional source or error when using wheel odometry for dead-reckoning estimates, as does the skid-steer drive of the robot. For these reasons the wheel odometry was regarded as highly unreliable. The MASS relies on a number of alternate sensors and algorithms to estimate its pose since the software designed for the MASS treats dead-reckoning as an unacceptable source for localization.\\
\begin{figure}[htb]
    \centering
    \includegraphics[width=.8\textwidth]{Pics/sfeats.jpg}
    \caption{Mock Mine Surface Features}
    \label{fig:surfacefeature}
\end{figure}
\clearpage
\section{Test Plan}
\label{sec:scenario}
\subsection{Localization}
\label{see:loctestscenario}
The MASS has a number of options for verifying its fiducial based location estimation algorithm. In order of increasing accuracy, they are:
\begin{itemize}
    \item Odometry
    \item Inertial Measurement
    \item Laser Scan Matching
    \item Simultaneous Localization and Mapping
    \item Adaptive Monte Carlo Localization
    \item OptiTrack Motion Capture
\end{itemize}

Odometry is measured using encoders mounted along the drive-train of the wheels. The main source of error in odometry readings is from wheel slippage, but there is no way to detect if the wheels are slipping from the encoder readings alone. Inertial measurement units (\acrshort{imu}) use accelerometers, gyroscopes, and sometimes magnetometers to estimate relative motions of the unit. Accelerometers can estimate position by integrating the acceleration data twice to produce position data, but the effect of noise in the signal is amplified with each integration. Gyroscopes can measure angular acceleration, but the data and its accompanying noise must also be integrated twice to estimate orientation. A magnetometer can provide an absolute reference using the Earth's magnetic field, but must be calibrated for its environment and performs poorly in the presence of electronic interference or ferrous materials. Since there is no absolute reference, all position estimates are made relative to the previous readings. This means any error introduced in an estimate is added to all the errors of the previous estimates and the cumulative error can grow unbounded. Dead-reckoning is generally regarded as a low accuracy form of localization. Since the MASS is a skid-steered vehicle, the odometry estimates are particularly unreliable due to the slippage that occurs when turning. Dead-reckoning estimation is most accurate when odometry and inertial measurement data are fused using a Kalman filter, however, without an absolute measurement of the system, it is impossible to provide a sufficiently accurate position estimate to use as a ground truth for comparison to the fiducial based location estimation algorithm.\\

Using the \acrshort{lidar} scanner on the robot, odometry can be estimated by matching laser scans and calculating the transformation between them. Laser scan matching does not lose accuracy due to wheel slippage since it is measuring distances to fixed surfaces in its environment, but it still only produces relative position estimates and as such its error can grow unbounded.\\

Simultaneous Localization and Mapping (\acrshort{slam}) uses the \acrshort{lidar} scanner to build a map of the environment and localize itself within the map. Since it uses fixed points of reference, the error remains bounded. While some \acrshort{slam} algorithms employ Adaptive Monte Carlo Localization (\acrshort{amcl}) to localize within the generated map, it is possible to generate a highly accurate map prior to testing and only use \acrshort{amcl} for localization. Since a previously generated map can be post-processed to optimize accuracy and reduce errors, \acrshort{amcl} alone has been placed higher than \acrshort{slam} for localization estimation accuracy.\\

The most accurate localization system available for this research is the OptiTrack motion capture system. The manufacturer states the system regularly achieves a positional error less than 0.3 mm and rotational error below 0.05\degree. Since the motion capture system is an absolute measurement system, any errors in previous measurements have no effect on the current position estimate. The OptiTrack system uses an array of cameras optimized for detecting infrared (IR) light. Each camera has IR \acrshort{led}s to increase the amount of infrared light available for reflection. To use the tracking system, IR markers are placed on the robot to be tracked. The IR markers can be active or passive. Passive markers are coated with a material that is highly reflective to infrared light and can be easily seen by the cameras. Active markers contain high powered IR \acrshort{led}s that emit IR light. Each active marker blinks at a specific frequency so that it can be uniquely identified by the tracking system. A single marker can only provide position information, so three markers are combined to form a trackable object whose pose can be determined. The trackable object is taught to the system using the Motive tracking tools software. Three or more markers are selected and the distance between each of the markers form the parameters of the trackable object. Figure \ref{fig:trackable} shows one of OptiTrack's rigid body trackable objects formed using one of OptiTrack's rigid bodies and three reflective markers. Figure \ref{fig:motivetrackable} shows the three reflective markers selected within the Motive software to form the trackable object. IR reflective markers are placed on three or more of the six arms of the rigid body. Using between three and six markers on the six arms of the rigid body, 28 unique configurations can be assembled. If more than 28 objects need to be tracked at a time, alternate rigid body shapes or active markers are required.\\

\begin{figure}
    \centering
    \includegraphics[width=.6\textwidth]{Pics/rigdi.jpg}
    \caption{Rigid Body Trackable Object With Three IR Markers Installed}
    \label{fig:trackable}
\end{figure}
\begin{figure}
    \centering
    \includegraphics[width=\textwidth]{Pics/ss3.png}
    \caption{OptiTrack Interface, Trackable Object Selected}
    \label{fig:motivetrackable}
\end{figure}

The OptiTrack cameras are placed in the robot environment such that the field of view of the cameras cover as much of the workspace as possible, with as much overlap between the cameras' field of view as possible. Once the cameras are placed in their fixed locations, the system can be calibrated. Calibration is achieved using the calibration wand shown in Figure \ref{fig:wand}. Having specific spacing between the collinear markers allows the tracking software to recognize the calibration wand when in view of the cameras. The operator moves the wand through as much of each camera's field of view as possible. The software records the wand trajectory seen by each camera until the wanding process is complete. Once finished, the software computes each camera's location and image distortion in an iterative process. The longer this process runs for, the more accurate the results are. Once the system determines each camera's location and image distortion, the system is calibrated and ready for use. The origin and ground plane is set using the ground plane tool shown in Figure \ref{fig:groundplane}. It is placed level with the ground at a point the operator wishes to define as the origin for OptiTrack's coordinate system.\\

\begin{figure}
    \centering
    \includegraphics[width=.4\textwidth]{Pics/calwand.jpg}
    \caption{OptiTrack Calibration Wand \cite{caltools}}
    \label{fig:wand}
\end{figure}
\begin{figure}
    \centering
    \includegraphics[width=.4\textwidth]{Pics/calsquare.jpg}
    \caption{OptiTrack Ground Plane Tool \cite{caltools}}
    \label{fig:groundplane}
\end{figure}

Since the OptiTrack motion capture system is the most reliable localization system available to track the location of the MASS, it was chosen as the ground truth for the system. Ground truth is the positional data that is considered the most accurate value available. The accuracy of the MASS localization system is determined by comparing the location estimates generated by the system to the ground truth values. The origin of the world coordinate frame generated by the MASS is set by the user when they perform the initial localization scan so it may not be coincident with the coordinate frame of the OptiTrack system. To test the accuracy of the localization system, the offset between the MASS coordinate frame origin and the OptiTrack coordinate frame origin was determined by recording the MASS location measured by OptiTrack at the time a homing scan was performed.\\

\subsubsection{Localization Test}

The localization test was intended to verify the accuracy of the localization algorithm. The localization algorithm must detect the fiducial marker and determine its location. To do so, it uses the \acrshort{lidar} and nodding head to create a point cloud of the environment, detect the position and orientation of the fiducial marker, and calculate the coordinate frame transformation from the \acrshort{lidar} to the marker. The inverse of the transformation indicates where the scan was taken from. The aim of this test is to confirm the algorithm is capable of providing localization estimates within 5 cm of the ground truth location.\\

\begin{figure}
    \centering
    \includegraphics[width=\textwidth]{Pics/20180303_174007.jpg}
    \caption{MASS With Trackable Rigid Body Marker}
    \label{fig:rigidbody}
\end{figure}

The OptiTrack motion capture system was used as ground truth for the robot. The MASS can be seen with the rigid body marker in Figure \ref{fig:rigidbody}. The marker was taught to the OptiTrack software as a trackabale object.\\

When the system performs a localization scan, it detects and reports the location of the fiducial marker. The transformation matrix from the robot base (B) to the fiducial marker centre (M) is defined as $^B_M\mathbf{T}$. To verify the accuracy of the localization system the matrix $^B_{M}\mathbf{T}$ is computed and the location of the marker, $^B_{M}\mathbf{p}$, is determined using the following properties of a transformation matrix:\\

\begin{equation*}
\mathbf{T} = \begin{bmatrix}
    r_{11} & r_{12} & r_{13} & p_{x} \\
    r_{21} & r_{22} & r_{23} & p_{y} \\
    r_{31} & r_{32} & r_{33} & p_{z} \\
    0 & 0 & 0 & 1
\end{bmatrix}
\end{equation*}

The location of the robot with respect to the origin of the world coordinate frame defined during the homing scan is determined using the following equation:

\begin{equation*}
^O_B\mathbf{p} = ^O_M\mathbf{p} - ^B_M\mathbf{p}
\end{equation*}

with $O$ representing the world coordinate frame origin, $M$ the marker location (with respect to the robot base), and $B$ the position of the robot base.\\

\begin{figure}
    \centering
    \includegraphics[width=.8\textwidth]{Pics/scanlocations.png}
    \caption{Localization Testing Scan Locations}
    \label{fig:scanlocations}
\end{figure}

The test was performed 50 times at different locations, as indicated in Figure \ref{fig:scanlocations}. The locations were chosen by driving the robot around the mock mine area and taking scans at regular intervals. The size of the MASS restricts access to locations with close proximity to the mine surface. Seven marker keypoints were placed on the mine and their positions saved to a marker file as five individual markers, allowing the localization algorithm to select the marker which it can most accurately locate when performing localization.\\

\subsubsection{Localization Test Results}
\label{sec:locresults}
\begin{center}
\begin{table}[ht!]
\small
\resizebox{\textwidth}{!} {
\begin{tabular}{|c|c|c|c|c||c|c|c|c|c|}
\hline
\multicolumn{2}{|c|}{Localized Position} & \multicolumn{2}{c|}{Ground Truth} & Error & \multicolumn{2}{c|}{Localized Position} & \multicolumn{2}{c|}{Ground Truth} & Error\\ \hline
X(m) & Y(m) & X(m) & Y(m) & (m) & X(m) & Y(m) & X(m) & Y(m) & (m) \\ \hline \hline
0.7384 & -1.3880 & 0.7244 & -1.3846 & 0.0072 & 0.5060 & 1.0610 & 0.4608 & 1.0915 & 0.0380 \\ \hline
-0.4010 & -1.5420 & -0.4216 & -1.5353 & 0.0086 & -0.0154 & -1.4604 & -0.0078 & -1.4172 & 0.0386 \\ \hline
-0.9384 & -1.3487 & -0.9471 & -1.3284 & 0.0106 & 0.3848 & 1.2865 & 0.3340 & 1.2860 & 0.0395 \\ \hline
-0.6185 & 0.0380 & -0.6352 & 0.0372 & 0.0119 & -1.4465 & -1.3406 & -1.4982 & -1.3213 & 0.0398 \\ \hline
-1.4818 & -1.3196 & -1.4975 & -1.3211 & 0.0124 & 0.0224 & -1.4635 & -0.0074 & -1.4162 & 0.0404 \\ \hline
-0.3920 & -1.5397 & -0.4196 & -1.5343 & 0.0156 & 0.5774 & 0.9583 & 0.5228 & 0.9667 & 0.0418 \\ \hline
0.4421 & 0.0111 & 0.4196 & 0.0352 & 0.0166 & -0.2736 & 1.0389 & -0.3271 & 1.0646 & 0.0434 \\ \hline
0.4386 & 0.0065 & 0.4197 & 0.0360 & 0.0198 & -0.8033 & 0.8371 & -0.8401 & 0.8845 & 0.0439 \\ \hline
-0.9400 & -1.3573 & -0.9468 & -1.3275 & 0.0202 & 0.5704 & 0.9631 & 0.5251 & 0.9438 & 0.0440 \\ \hline
1.4172 & 1.5971 & 1.4185 & 1.5930 & 0.0204 & 0.3845 & 1.2982 & 0.3344 & 1.2850 & 0.0441 \\ \hline
0.7192 & -1.3828 & 0.7249 & -1.3843 & 0.0222 & -1.3545 & 1.2250 & -1.3406 & 1.2709 & 0.0444 \\ \hline
0.8888 & 0.0411 & 0.8669 & 0.0751 & 0.0251 & 0.2809 & 1.6782 & 0.2236 & 1.6918 & 0.0445 \\ \hline
0.8869 & 0.0414 & 0.8648 & 0.0755 & 0.0252 & -0.1324 & -0.0390 & -0.1698 & 0.0088 & 0.0446 \\ \hline
1.0642 & 1.1003 & 1.0299 & 1.0968 & 0.0256 & 1.5628 & -0.0435 & 1.5458 & 0.0144 & 0.0476 \\ \hline
1.4259 & 0.4453 & 1.4324 & 0.4366 & 0.0273 & -0.1226 & -0.0363 & -0.1715 & 0.0071 & 0.0488 \\ \hline
-0.8060 & 0.8516 & -0.8400 & 0.8806 & 0.0280 & -0.0921 & -0.0222 & -0.1275 & 0.0319 & 0.0491 \\ \hline
1.4119 & 1.5968 & 1.4204 & 1.5878 & 0.0289 & -0.6706 & 0.0491 & -0.6382 & 0.0396 & 0.0495 \\ \hline
1.3574 & -0.2296 & 1.3584 & -0.2450 & 0.0295 & 0.5087 & 1.0540 & 0.4562 & 1.1042 & 0.0561 \\ \hline
-0.1549 & 1.1922 & -0.1911 & 1.2217 & 0.0300 & -0.0785 & -0.0263 & -0.1264 & 0.0296 & 0.0573 \\ \hline
-1.3592 & 1.2650 & -1.3407 & 1.2715 & 0.0316 & -0.6441 & 0.9206 & -0.6273 & 0.8752 & 0.0633 \\ \hline
0.8002 & 1.0567 & 0.7600 & 1.0864 & 0.0334 & -1.1577 & 0.2472 & -1.1730 & 0.3260 & 0.0684 \\ \hline
-0.2748 & 1.0372 & -0.3208 & 1.0525 & 0.0335 & 0.7214 & 1.6581 & 0.6426 & 1.6960 & 0.0713 \\ \hline
1.5598 & -0.0415 & 1.5453 & 0.0052 & 0.0362 & -0.6479 & 0.9212 & -0.6223 & 0.8689 & 0.0736 \\ \hline
1.0792 & 1.0974 & 1.0301 & 1.1037 & 0.0364 & 1.5845 & 1.0444 & 1.5201 & 1.1090 & 0.0747 \\ \hline
0.8028 & 1.0619 & 0.7575 & 1.0908 & 0.0372 & 1.5807 & 1.0336 & 1.5210 & 1.1200 & 0.0892 \\
\hline
\end{tabular}
}
\caption{Localization Testing Results}
\label{tab:optiresults}
\hfill{}
\end{table}
\end{center}

Defining the error as the absolute distance between the localization system's position estimate and ground truth, the mean error was 37.8 mm with a standard deviation of 18.3 mm. Shown visually, the distribution of error across the testing data can be seen in Figure \ref{fig:locerrordist}. Table \ref{tab:optiresults} shows each localization scan and its respective accuracy. \\

\begin{figure}
    \centering
    \includegraphics[width=0.5\textwidth]{Pics/LocHistogram.png}
    \caption{Histogram of Localization Errors}
    \label{fig:locerrordist}
\end{figure}

\subsubsection{Localization Sources of Error}
\label{sec:locsourceerror}

The ground truth determined by the OptiTrack system is the first significant source of error. Though the manufacturer states a typical accuracy of 0.3 mm, and the calibration at the time of testing reported similar accuracy, the trackable marker mounted on the MASS was not tracked to that level of accuracy. When multiple scans were taken at a single location it was found the OptiTrack position estimate would vary an average of 9.2 mm. This error is most likely due to the tracking marker used with the MASS. Since the  tracking marker is the passive type, the marker may appear distorted relative to the way it appears in other areas of the system's field of view. This distortion is increased when the trackable marker is located near the perimeter of the tracking area or when a portion of the marker is partially occluded. The use of active markers will yield more accurate ground truth measurements.\\

\begin{figure}
    \centering
    \includegraphics[width=0.5\textwidth]{Pics/RepeatabilityHistogram.png}
    \caption{Histogram of Repeatability Errors}
    \label{fig:repeathist}
\end{figure}

Another limiting factor in the accuracy of the localization estimate is the accuracy of the \acrshort{lidar} scanner. The manufacturer states a systematic error of $\pm$30 mm and statistical error of 12 mm, however, the accuracy decreases with larger distances and less reflected light. Boehler and Marbs provide a comprehensive analysis of the sources of error in laser measurement devices \cite{checkthebookmarksbar}. To determine the experimental accuracy of the \acrshort{lidar} device, repeatability testing was performed in the \acrshort{mars} lab mock mine. Accuracy testing is difficult to achieve, but can be performed by physically verifying distance measurements using a measuring tape or placing markers detectable by both the \acrshort{lidar} and OptiTrack system and comparing the measured distances. Since these techniques introduce either human or measurement error, and algorithms for thickness estimation and localization require repeatability more than accuracy, the focus was to determine the repeatability of the \acrshort{lidar} scanner more so than the accuracy. By performing 12 scans at the same location, and comparing the distance from each scan's point to a mesh generated by the combined point clouds it was found the \acrshort{lidar}'s repeatability had a standard deviation of 6.7 mm as shown in Figure \ref{fig:repeathist}. The the reported accuracy meets the requirements but the true accuracy of the localization system was in fact better than the reported accuracy. Using the localization system to register two point clouds, the thickness estimation algorithm can be used on portions that remain unchanged. The resulting thickness estimates represent the error in localization combined with the measurement error of the \acrshort{lidar}. Similar to the scenario discussed in section \ref{sec:placing}, the reported accuracy is proportional to the distance from the scanner to the location within a point cloud, making it difficult to provide a true estimate of the localization accuracy. Since the purpose of the localization system is to accurately register point cloud scans for thickness estimation, OptiTrack was only used as a form of verification for localization accuracy.\\
\begin{figure}[hb]
    \centering
    \includegraphics[width=.8\textwidth]{Pics/bumpsinset.png}
    \caption{High Reflectivity Areas Creating False Contours (Top Image Shows Reflected Intensity)}
    \label{fig:bumps}
\end{figure}
\clearpage
For determining the accuracy of the point-to-mesh (\acrshort{p2m}) thickness estimation technique (in which the distance is calculated normal to the mesh surface), infrared (IR) reflective stickers were placed on the simulated shotcrete and mine surface such that they are collinear along a normal to the simulated shotcrete surface. When the \acrshort{lidar} scans were analyzed, it became apparent how the reflectivity of the surface affects the distance measurement error as shown in Figure \ref{fig:bumps}. It was found the false contours protruded from the surface approximately 20 mm.\\

The \acrshort{lidar} scanner is mounted on a servo motor acting as a nodding head. The servo is a SCHUNK PR70 rotary module having a 0.08\degree\hspace{0pt} repeatability. Since the encoder is relative, there is no absolute accuracy specifications provided by the manufacturer. It uses end stops and a homing routine to determine its position when brought online. When the nodding head moves the \acrshort{lidar} scanner, each individual laser scan is combined to form a point cloud. If the nodding head reports an incorrect position, the laser scan will be positioned incorrectly in the point cloud. Nodding head position error can come from the servo itself, or occur due to software latency. Since the operating system is not designed as a real-time operating system, the servo position data may get delayed when being sent to the laser scan assembler. The \acrshort{lidar}, DENSO controller, and WiFi connection require the MASS to have an on-board router. For teleoperation and communication with the OptiTrack system the on board router must connect to a wireless network. The network is not configured to prioritize \acrshort{lidar} data, presenting an additional source of latency. Figure \ref{fig:headlag} shows a point cloud where the laser scan data had significant latency. Though all non-essential processes are halted during the scanning process, due to the current hardware limitations of the MASS many scans had latency errors.\\
\clearpage
\begin{figure}
    \centering
    \includegraphics[width=.8\textwidth]{Pics/lag.png}
    \caption{Point Cloud Data With Visible Gaps Due To Latency}
    \label{fig:headlag}
\end{figure}
The mounting bracket for the laser scanner was designed to flex as little as possible. If the mounting bracket flexes or vibrates, the calculated position of the \acrshort{lidar} will differ from the true position. Since the laser scan assembler uses the forward displacement solution to calculate the nodding head position and cannot detect mechanical vibration or deformation, these sources of error can cause inaccuracy in the assembled point clouds. With this potential for error in mind, the mounting bracket was designed to allow as little vibration and deformation as possible.\\

The \acrshort{lidar} scanner measures a single point at a time and uses an internal rotating mirror to sweep the laser in a horizontal line at a frequency of 25 Hz. If the nodding head moves fast enough the position of the first point in the scan will be on a different plane than the last point in the scan. If the nodding head transmits the servo position at a higher frequency than the \acrshort{lidar} scans, the laser scan assembler can transform each individual point measurement into the fixed coordinate frame rather than the whole scan at once. This is a fairly CPU intensive process and it was determined that a sufficiently slow slew rate could achieve accurate results. The maximum skew is defined as:

\begin{align}
    S_{max} = tan\left(\frac{\omega_H}{f_L}\right)
\end{align}

where $S_{max}$ is the maximum skew angle from the first point in the scan to the last, $\omega_H$ is the nodding head angular velocity, and $f_L$ is the scanning frequency of the \acrshort{lidar}.\\

\begin{figure}[h]
    \centering
    \includegraphics[width=.8\textwidth]{Pics/edgeveiling.png}
    \caption{Edge Veiling Effect}
    \label{fig:veiling}
\end{figure}

One source of error easily visible to the human eye is the edge veiling effect and can be seen in Figure \ref{fig:veiling}. Since the laser beam is not a single point but a ray that spreads over distance, when it lands on an edge, it is possible that the area of measurement may only partially lie on a surface. The remainder of the measurement area may lie on a further object or beyond the measurement range of the scanner. The area that lands on the object's edge could be anywhere from a tiny fraction to a large majority of the measurement area. Depending on how much of the measurement area lies on the edge, how far the remaining measurement area is from the edge, and what the reflectivity of the surfaces are, the measured point will be reported as somewhere between the edge and the surface beyond. The edge effect can be minimized using the Intensity Recovery Algorithm presented in \cite{bookbaredge} and \cite{theotherone}. The \node{laser_filters} package for \acrshort{ros} provides a \func{ScanShadowsFilter} function that attempts to minimize the edge effect using the angle between the laser origin and two points. Given $P_1$ and $P_2$ with an origin $O$, the user can specify if $\angle OP_1P_2$ is beyond an acceptable range to remove the specified number of neighbouring data points. Since the fiducial marker keypoints used in this research are mounted such that they stand offset from the surface, it is possible the edge effect may affect the measured location of the keypoint.\\

\begin{figure}[h]
    \centering
    \includegraphics[width=\textwidth]{Pics/eddt.pdf}
    \caption{Two Sections of a Point Cloud Containing Keypoints. Left: Keypoint Near the Scanner. Right: Keypoint Far From Scanner}
    \label{fig:keypointresolution}
\end{figure}

Though the edge effect can lower the accuracy of the measured location of the fiducial marker keypoint, the resolution of the point cloud has a far greater impact. Figure \ref{fig:keypointresolution} shows two sections of a point cloud containing marker keypoints. Since the resolution and reflected intensity decreases over distance, the marker keypoints further away from the scanner will have less data points and a lower intensity. As well, the edge effect occurs not just to discontinuous surfaces, but also to discontinuities in the reflectivity of the surface. At further distances from the scanner, the laser point measurement area is larger and a significant enough portion may overlap the highly reflective surface of the marker keypoint to return a measured point with high enough intensity to be considered a portion of the marker keypoint. Figure \ref{fig:keypointinaccuracy} shows a portion of a point cloud containing a keypoint taken from a distance and highlights the possible area in which the keypoint may be. Since it is impossible to determine exactly where the keypoint is from that data, an averaging algorithm takes the mean location to be the assumed location of the keypoint.\\ 

\begin{figure}[h]
    \centering
    \includegraphics[width=.7\textwidth]{Pics/badpt.pdf}
    \caption{Keypoint Detected at a Distance}
    \label{fig:keypointinaccuracy}
\end{figure}
The software algorithm for localization has a minimal loss of accuracy. Occasionally the operating system must perform mathematical operations using values with more digits than a variable can hold. For example, $\pi$ is infinite and non-repeating so it is impossible to perform any calculations with a truly accurate value of $\pi$. However, the computer can perform operations with sufficient accuracy that rounding errors are likely not a significant source of error.\\

\subsection{Trajectory Generation}

The testing of the trajectory generation algorithm is difficult to quantify. Throughout development, visual confirmation was the main criteria for verifying successful trajectory generation. Simply said, if it was unsuccessful it would be immediately apparent. As the development process continued, the algorithm demonstrated consistently suitable trajectories. The criteria for failure are:

\begin{itemize}
    \item No trajectory generated
    \item Trajectory point outside robot workspace
    \item Trajectory does not maintain optimal distance from surface
    \item Trajectory does not cover the entire selected surface
    \item Trajectory path has undesirable overlap
\end{itemize}

As long as none of the failure criteria are met, the trajectory generation was considered successful. The testing strategy consists of generating trajectories from a variety of positions and orientations relative to the mine surface, as well as selecting different sections of the mine. Trajectories are generated at positions near and far from the walls, with the robot positioned parallel or perpendicular to the surface, as well as on wall sections, roof sections, and combined sections. Straight sections of the mine as well as curved sections were selected for trajectory generation. Figures \ref{fig:traj1}, \ref{fig:traj2}, and \ref{fig:traj3} show trajectories generated for various sections of the mine. For debugging purposes, the surface path the manipulator is to follow is shown in red, the actual path the end-effector follows is in green, and the normal vector of length equal to the desired offset is shown at each via point in blue. The lines that are used to calculate distance measured along the surface is shown with points ranging in colour from red to pink, with colour representing distance measured along the mine surface (or in the case of wall sections, simply the Z-height from the ground).\\

\begin{figure}[h]
    \centering
    \includegraphics[width=.8\textwidth]{Pics/front_combo.png}
    \caption{Trajectory Generation for a Wall Section}
    \label{fig:traj1}
\end{figure}
\begin{figure}
    \centering
    \includegraphics[width=\textwidth]{Pics/roof_combo.png}
    \caption{Trajectory Generation for a Wall and Roof Section}
    \label{fig:traj2}
\end{figure}
\begin{figure}
    \centering
    \includegraphics[width=\textwidth]{Pics/corner_combo.png}
    \caption{Trajectory Generation for a Flat, Curved, and Overhung Section}
    \label{fig:traj3}
\end{figure}
\clearpage

In the following sections, the causes of the failure criteria are discussed.

\subsubsection{No Trajectory Generated}

If the user commands the robot to spray an empty area, an empty point cloud is sent to the trajectory generation service and no trajectory is generated. Similarly, if the robot is not close enough to the mine surface and there exists no surfaces in the automatic shotcrete zone, or the automatic shotcrete zone is positioned such that it does not intersect with the mine surface, the trajectory service will not return a trajectory. In any of these scenarios, it is entirely operator error that caused the failure. To avoid such failures properly trained operators or autonomous functions and default values can be used.\\

There is a verification in the \node{control_panel} node to ensure it does not execute an empty trajectory. The result of this type of failure would be no action on the robot, a desired behaviour.

\subsubsection{Trajectory Point Outside Robot Workspace}

The trajectory generation service was designed to be as modular as possible. None of the restrictions specific to the MASS manipulator have been applied to the algorithm. The intention is that the service can be used in future work and on alternately configured systems. It is up to the user to ensure trajectory points are within the manipulator's workspace when using the trajectory generation algorithm as-is. In the MASS, when executing a trajectory, each point is verified to be within the robot workspace. If a trajectory point lies outside the robot workspace, the \node{control_panel} node will move the trajectory point to the nearest location within the manipulator's workspace. This decision was made to allow the MASS to demonstrate its trajectory generation and execution capabilities without workspace limitations. Since the MASS will move trajectory points that are not within the robot's workspace, a trajectory point outside the workspace does not result in an error in trajectory execution.\\

Additional causes of trajectory points lying outside the robot workspace is erroneous point cloud data. If the laser scanner detects a portion of the MASS such as the manipulator when generating point clouds, the trajectory generation algorithm will attempt to generate a trajectory that includes those points as part of the surface. To avoid false readings of the environment, the manipulator is moved outside the field of view of the \acrshort{lidar} when it is performing a surface scan. As well, the resulting data is cropped to remove any points that may result from the robot detecting portions of itself.\\

Without relocating the trajectory points, the manipulator will halt operation if commanded to move to a position outside its workspace. Since the \node{control_panel} node should automatically move trajectory points if necessary, halting operation is the desired behaviour if a trajectory point still exists outside the manipulator's workspace.\\

In future work, additional verification can be done by setting a minimum number of neighbouring points that must be found within a given distance of a via point, or the via point is ignored. Since this type of failure rarely occurred, the additional computation was deemed unnecessary.\\

\subsubsection{Trajectory Does Not Maintain Optimal Distance From Surface}

Though normally this would result in a failure of the manipulator to reach the desired via point, due to the workspace limitations of the MASS this scenario happens often due to moving the via point to fit within the manipulator's workspace. The only way to avoid this scenario with the MASS is to navigate very close to the mine surface and apply shotcrete to small sections at a time. A robot designed to operate in an actual mine will require a manipulator with a sufficiently large workspace. Since the algorithms developed in this work was tested using the current workspace limitations, and this configuration of the MASS is not designed to apply actual shotcrete, moving the trajectory points such that they are not properly offset from the mine surface is not considered a failure of the trajectory generation algorithm. When generating and displaying shotcrete trajectories, optimal distance was consistently achieved. It is only when executing the trajectory on the MASS that non-optimal distances from the mine surface may occur.\\

Due to the robust algorithm developed, the trajectory points will only be created at optimal distances from the mine surface. The only possibility of generating a misplaced trajectory point is if the Point Cloud Library (\acrshort{pcl}) normal estimation function fails unexpectedly. This has never occurred during testing or development and verifying the reliability of \acrshort{pcl} algorithms is outside the scope of this work. During execution, the trajectory points may be moved from outside the manipulator's workspace to within, which would result in sub-optimal distance from the mine surface. Being the intended action of the system, this scenario of sub-optimal spray distance is not considered a failure.\\

\subsubsection{Trajectory Does Not Cover The Entire Selected Surface}

The trajectory algorithm uses a subtractive approach to generating a shotcrete application trajectory. As it generates trajectory points it removes the original points from the point cloud passed to it. Using this approach the entire surface will be covered by the generated trajectory.\\

If a section of the surface does not receive shotcrete, the thickness estimation will detect the area has insufficient shotcrete and the system can autonomously navigate to that area and reapply shotcrete. Without the additional verification of shotcrete thickness performed by the MASS, not covering the intended surface with shotcrete could be a significant failure resulting in risk to humans present. It is therefore recommended that the thickness estimation service be used to verify successful shotcrete application.
\subsubsection{Trajectory Path Has Undesirable Overlap}

A certain amount of overlap may be desirable when applying shotcrete. This can be configured using the \node{dynamic_reconfigure} node, \acrshort{ros} parameter server (through command line utilities), or changing the configuration file for the MASS. An undesirable overlap occurs if the trajectory passes over a previously shotcreted area. Since the trajectory generation algorithm removes the points it has generated a spray trajectory for, it is unlikely the system can produce a trajectory that overlaps itself. During testing and development, the trajectory generation algorithm never produced overlapping shotcrete spray paths.\\

Should the algorithm be modified in a way that results in undesirable overlapping spray patterns, additional shotcrete will be wasted. As well, applying too thick of a shotcrete layer before allowing it to dry may cause shagging due to loss of adhesion. Overlapping dry layers of shotcrete does not reduce the safety provided by the shotcrete layer and the material wastage is minimal. In the absence of shagging this failure is not critical to human safety or mine operation, but should the shotcrete material not remain where it was applied the thickness estimation algorithm is capable of detecting it and reapplying as necessary.\\

\subsection{Thickness Estimation}

\begin{figure}[h]
    \centering
    \includegraphics[width=.9\textwidth]{Pics/controlpt.pdf}
    \caption{Control Points Marked on Mine and Simulated Shotcrete}
    \label{fig:controlpts}
\end{figure}

To estimate the thickness of the applied shotcrete, an artificial shotcrete is applied to the surface of the mine. Magnets are used to ensure consistent placement of the artificial shotcrete. Control points were placed on the surface of the mine as well as the surface of the artificial shotcrete as shown in Figure \ref{fig:controlpts}. The control points are highly reflective to infrared light, so they are easy to distinguish from the surrounding area when examining Optitrack or \acrshort{lidar} point cloud data. The control points were placed normal to the artificial shotcrete surface. Using the control points, the point-to-mesh (\acrshort{p2m}) thickness estimation can be verified, however the point-to-point (\acrshort{p2p}) approach would require different placement to verify. As discussed in Section \ref{sec:locsourceerror} and shown in Figure \ref{fig:bumps}, the control points became a source of error rather than a verification tool.\\

Accuracy is verified through the use of control measurements. By taking scans before and after the artificial shotcrete is applied without moving the MASS, thickness estimates can be generated without introducing the localization algorithm as a source of error. Though the thickness estimates using the \acrshort{p2p} approach cannot be physically verified, the high repeatability of the \acrshort{lidar} scans can be used to produce a reasonably accurate control measurement. To generate an accurate estimate of the thickness of the artificial shotcrete layer, 12 scans were taken from a single location of the bare surface, and another 12 were taken from the same location after applying the artificial shotcrete. The 12 bare surface scans were combined to a single point cloud, then processed to remove outliers and reduce the density using the CloudCompare software. The downsampled point cloud was then converted to a mesh representation using Poisson surface reconstruction to form a continuous surface. The initial mesh had much more surface roughness than the true surface due to sensor noise, so surface smoothing was performed to improve the quality of the mesh representation. The 12 post-shotcrete scans were also combined but not downsampled. The combined post-shotcrete point cloud was compared to the pre-shotcrete surface mesh using CloudCompare's \acrshort{p2m} tool to produce a high density point cloud of thickness measurements to which other thickness estimation point clouds can be compared to, herein referred to as the control cloud.

Thickness estimation was performed on the same section of artificial shotcrete from a variety of locations. Since the accuracy of the localization was tested separately, the most accurate fiducial marker configuration was chosen. If using a similar configuration for the marker is not possible, the accuracy confidence can be lowered accordingly for the shotcrete thickness estimation. The robot was positioned at multiple locations, both near and far from the region of interest. The locations of the scans can be seen in Figure \ref{fig:thickscans}.\\

\begin{figure}[h]
    \centering
    \includegraphics[width=.8\textwidth]{Pics/postprescanlocations.png}
    \caption{Thickness Estimation Scan Locations (Pre-Shotcrete in Green, Post-Shotcrete in Red)}
    \label{fig:thickscans}
\end{figure}

There were 19 locations chosen as pre-shotcrete scans, and 12 locations chosen as post-shotcrete scans. Comparing each post-shotcrete scan to every pre-shotcrete scan yielded 228 thickness estimation point clouds. The point clouds were compared to the control cloud and the measurement error was computed at each point. The mean error of each cloud was computed to give each cloud a single error measurement. The mean error of all the clouds was 23.3 mm with a standard deviation of 13.7 mm and a histogram of the data can be seen in Figure \ref{fig:thist}. Table \ref{tab:thickres} shows the mean error of each point cloud along with the number of data points per cloud. Figure \ref{fig:thickeg} shows a typical thickness estimate point cloud.\\

\begin{figure}[h]
    \centering
    \includegraphics[width=0.4\textwidth]{Pics/thickeg.png}
    \caption{Thickness Estimates Presented as Point Cloud}
    \label{fig:thickeg}
\end{figure}
\begin{figure}[h]
    \centering
    \includegraphics[width=0.5\textwidth]{Pics/THistogram.png}
    \caption{Histogram of Thickness Estimation Errors (m)}
    \label{fig:thist}
\end{figure}
\clearpage
\begin{table}[]
\begin{tabular}{cc|c|c|c|c|c|c|c|c|c|c|c|c|}
\cline{3-14}
\multicolumn{1}{l}{}                         &        & \multicolumn{12}{c|}{Post Shotcrete}                        \\ \cline{3-14} 
\multicolumn{1}{l}{}                         &        & 1 & 2 & 3 & 4 & 5 & 6 & 7 & 8 & 9 & 10 & 11 & 12 \\ \cline{3-14} 
\multicolumn{2}{c|}{\# Pts} &1063&2021&1222&67968&1368&3448&2817&9719&521&1052&552&1303 \\ \hline
\multicolumn{1}{|c|}{\multirow{20}{*}{\begin{tabular}{@{}c@{}}\rotatebox[origin=c]{90}{Pre Shotcrete}\end{tabular}}} & 1      &27.6&29.9&50.3&61.0&39.1&44.6&43.3&54.8&38.6&16.7&40.6&34.0    \\ \cline{2-14} 
\multicolumn{1}{ |c| }{}&2&20.7&22.9&42.8&48.6&31.9&36.2&35.8&46.4&31.7&11.1&33.3&26.8\\ \cline{2-14}
\multicolumn{1}{ |c| }{}&3&15.5&16.5&30.6&38.6&21.3&26.3&24.5&35.1&19.7&15.6&21.9&19.0\\ \cline{2-14}
\multicolumn{1}{ |c| }{}&4&12.4&13.9&33.5&40.5&22.2&27.8&26.3&37.1&22.4&12.2&23.8&17.4\\ \cline{2-14}
\multicolumn{1}{ |c| }{}&5&13.5&12.9&13.0&22.3&8.7&13.3&9.7&17.2&7.2&26.1&7.5&11.7\\ \cline{2-14}
\multicolumn{1}{ |c| }{}&6&12.0&13.2&30.0&37.3&19.4&25.0&23.4&33.8&18.9&13.7&20.6&16.0\\ \cline{2-14}
\multicolumn{1}{ |c| }{}&7&12.7&12.7&18.7&25.5&11.8&17.0&14.4&22.4&9.7&21.5&11.2&13.4\\ \cline{2-14}
\multicolumn{1}{ |c| }{}&8&11.9&11.8&16.0&24.6&9.8&14.8&11.6&20.2&7.9&23.4&8.9&11.4\\ \cline{2-14}
\multicolumn{1}{ |c| }{}&9&10.4&10.2&17.9&25.7&9.8&14.7&12.1&22.0&10.1&21.2&9.9&10.1\\ \cline{2-14}
\multicolumn{1}{ |c| }{}&10&11.3&11.1&21.0&28.2&11.8&17.7&15.1&24.4&12.0&19.7&12.9&11.8\\ \cline{2-14}
\multicolumn{1}{ |c| }{}&11&12.0&12.2&22.2&31.7&13.3&19.0&16.8&26.0&11.8&18.9&13.7&13.7\\ \cline{2-14}
\multicolumn{1}{ |c| }{}&12&9.6&9.4&17.2&24.6&9.2&13.9&11.4&21.2&10.2&21.7&9.4&8.8\\ \cline{2-14}
\multicolumn{1}{ |c| }{}&13&15.4&13.8&9.5&17.1&7.6&9.7&7.2&12.9&8.4&28.9&7.0&10.8\\ \cline{2-14}
\multicolumn{1}{ |c| }{}&14&20.3&18.7&7.0&11.3&10.9&10.0&8.7&9.5&11.2&33.9&9.7&15.0\\ \cline{2-14}
\multicolumn{1}{ |c| }{}&15&29.9&32.1&52.6&61.8&41.4&46.0&45.4&57.0&40.7&17.5&42.9&36.3\\ \cline{2-14}
\multicolumn{1}{ |c| }{}&16&33.8&36.3&57.4&62.3&45.8&51.5&50.4&60.7&46.0&19.8&47.6&40.5\\ \cline{2-14}
\multicolumn{1}{ |c| }{}&17&14.5&16.0&33.9&43.3&23.2&28.6&27.1&38.0&22.2&12.8&24.3&19.3\\ \cline{2-14}
\multicolumn{1}{ |c| }{}&18&18.0&19.9&39.2&44.8&28.3&33.0&32.2&42.7&28.2&10.8&29.6&23.7\\ \cline{2-14}
\multicolumn{1}{ |c| }{}&19&32.9&31.2&13.1&10.8&23.0&18.4&19.7&10.4&22.6&44.8&21.7&27.6\\ \hline

\end{tabular}
\caption[Thickness Estimation Results]{Mean Thickness Estimation Error (in mm)}
\label{tab:thickres}
\end{table}

\subsubsection{Thickness Estimation Sources of Error}

Taking multiple scans at a single location for use as a control measurement removes the potential source of error from the localization algorithm, but factors like systematic measurement error, robot vibration or deformation, edge veiling, and other environmental conditions are still present. Any sources of error discussed in Section \ref{sec:locsourceerror} not dependant on the software algorithm is also a source of error in thickness estimation.\\

While the edge effect can be identified and corrected when the edge is far from the background surface, in the case of small rock outcroppings creating many small edges, the edge effect will appear to slightly smooth out the surface. The edge effect over small uneven surfaces will not have a large effect on the accuracy of thickness estimation. Since the edge effect effectively interpolates between the edge and the background, and proper application of shotcrete should yield a smooth surface, the main occurrence of this error would lie in the pre-shotcrete scans. However, since the edge effect would return a point further away from the surface toward the scanner's viewpoint than the true value, the error lands on the side of underestimating the thickness. As long as the thickness estimates are underestimated, safety assurances are maintained. Shotcrete usage and rebound calculations will be affected, but since they are not within the main scope of this work, the current level of accuracy has been deemed sufficient.\\

Though the control measurements are not subject to errors introduced by localization, the actual thickness measurements are. If the robot moves during shotcrete application, it must use its localization algorithm to register the two point clouds to a common coordinate frame. Therefore, the thickness estimates cannot achieve higher accuracy than the localization system. If the accuracy of the localization is not sufficient, the system can be configured to take scans before each time it has to move, allowing it to achieve equal accuracy to the control measurements provided it does not move during shotcrete application.\\

The control measurements as well as the thickness estimates are subject to errors caused by the thickness estimation algorithm. As discussed in Sections \ref{sec:p2p} and \ref{sec:p2m}, the \acrshort{p2p} and \acrshort{p2m} methods of estimating thickness each have their own unique sources of error.\\

\section{Chapter Summary}

The results from testing the localization, trajectory generation, and thickness estimation algorithms were presented. The localization testing yielded satisfactory results, despite the significant amount of error sources as well as verification error. Recommendations for future work will present strategies to greatly improve the results. Sample shotcrete application or radiation scan trajectories were shown for multiple sections of the mock mine. By moving the end-effector via points and aiming at the surface via point, the MASS was able to execute the generated trajectories without moving the base. Trajectories that require repositioning of the robot were tested, and was achieved using multiple single pose shotcrete spray trajectories. The thickness estimation results were also presented. Given the thickness estimation accuracy is dependent on the accuracy of the localization, the results were quite promising. Many additional steps can be taken during post-processing of the data if necessary, but the goals of being robust and easy to use lead the author to present the data as-is. If greater accuracy at the cost of increased expertise and post-processing is desired, the recommendations for future work describe how to do so.\\