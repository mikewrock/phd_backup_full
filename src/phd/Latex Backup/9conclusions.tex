\chapter{Conclusions and Recommendations for Future Work}
\label{chap:conclusions}
The mining industry is slow to adopt new technologies, but the testing results provided by the MASS shows the algorithms used hold great promise in automating aspects of the mining process. Not only can the MASS achieve greater consistency in the tasks it automates, but offers much better verification for the work it does. As well, it offers the benefit of removing the operator from a hazardous environment, allowing them to work remotely. Between 2012 and 2017, there were 17 deaths and 169 critical injuries in the Ontario mining industry as reported by the ministry of labour \cite{seebelow}. If this work could save a single life, the economic benefits would pale in comparison. Though worker safety is of utmost importance, a secondary goal of this work was to reduce the financial cost of mining, either through efficiency, reduced training time, or increased operational availability. A conclusive answer to whether these goals have been achieved cannot be given until the system has been deployed in a mine, but the research indicates that both worker safety and mine efficiency will increase with the use of the MASS or its algorithms.\\
%https://www.ontario.ca/document/mining-sector-plan-2017-2018/mining-enforcement-statistics

The fiducial marker localization algorithm was developed for use in underground mining where GPS is not available. Localization can be achieved without the use of fiducial markers, but installing them in the mine yields far greater accuracy than alternative methods. The fiducial marker keypoints were chosen for their highly reflective properties, allowing them to be reliably and consistently detected. Installing the IR reflective spheres that serve as marker keypoints is not necessary since there are many alternatives. IR reflective spray paint can be used in place of the spheres at minimal cost to accuracy. As shown in Figure \ref{fig:minepaint}, there is already an established practice of installing permanent markers in the mine for surveying purposes, and there is no issue or limitations with applying spray paint to the mine. Even without using IR reflective spray paint the LiDAR will easily detect the white paint shown. For greater accuracy solid IR reflective spheres can be attached in a removeable way to the permanently installed marker shown allowing re-use of the markers when localization in that area is no longer necessary. Using that approach keypoints can be placed on the surface that requires shotcrete as long as the metal marker protrudes far enough that it does not get covered by the shotcrete layer.\\

\begin{figure}
    \centering
\includegraphics[width=0.75\textwidth]{Pics/painted.jpg} 
    \caption{Permanent marker for mine surveying}
    \label{fig:minepaint}
\end{figure}

The localization algorithm can be used my much more than just the MASS, any autonomous vehicle in the mine can use the fiducial markers for localization. The markers can be used for loop closure, a common problem in SLAM algorithms. When a robot navigates using SLAM and drives in a loop the algorithm often has errors in the map when it completes the loop. Upon completion of the loop the algorithm must reconcile the difference between the environment it currently sees versus the map it previously built the last time it was in that location in a process called loop closure. Since SLAM generates position estimates based on previous information, errors can build over time but do not grow unbounded since the landmarks used are absolute references. Having a highly accurate absolute reference point allows for much more accurate loop closure.\\

The whole purpose of mining is to extract material from the earth. To do so, a large volume of material must be transported, and often requires processing. Transporting material from the removal site to the processing site is another task well suited for automation. To automate the transport of material, localization is necessary. Vehicles that autonomously transport material, equipment, or workers can benefit from the use of this localization algorithm.\\

The marker keypoints can be used for more than localization as well. Since every scan is localized to a common coordinate frame, the point clouds can simply be combined to produce a 3D model of the mine. This information is useful for visualization, planning, backfilling, and monitoring of the mine. Over time the weight of the earth compresses the mine in a process called rockmass convergence. By comparing the initial parameters of the marker with scans taken at a later date, the relative motion of the keypoints can be used to measure uneven rockmass convergence. If the keypoints are fixed in such a way they do not move as the rockmass converges, they can be used to measure the amount of rockmass convergence by generating a second 3D model of the mine at a later date and using the thickness estimation algorithm to detect and display the amount of change. Alternatively, keypoints can be placed in areas when rockmass convergence is not likely to occur, such as the entrance to the mine, and the system can determine how the keypoints have moved relative to the fixed keypoint locations.\\

The trajectory generation for shotcrete application and radiation scanning yielded consistent and satisfactory results. Some of the initial concepts seemed more elegant and robust, but in practice the simpler approach was more reliable and effective. The parameters of the spray pattern is easily changed, allowing users to apply the algorithm to many other applications. Alternate sensors or mobile manipulator systems can be used with algorithm to perform tasks like temperature mapping of a surface, acoustic sampling, or measuring the penetration of electromagnetic signals through a surface. The shotcrete and scanning tool changer developed at the MARS Laboratory can be replaced or modified allowing the trajectory generation algorithm to be applied to tasks like painting, coating, or washing a surface.\\

The thickness estimation algorithm implemented in ROS allows users a seamless experience when applying shotcrete and measuring its thickness. The framework built in this research allows for the possibility of implementing existing or novel point cloud comparison algorithms within the ROS environment. Though alternate point cloud comparison software is available, as of the time of publication none have been implemented in ROS other than those discussed herein. With the ability to compare previously saved point clouds or selected portions of them, the point cloud comparison algorithm can be used as-is for applications other than shotcrete thickness estimation like rockmass convergence, erosion calculations, or forest canopy growth. As long as two point clouds can be generated, they can be compared using the MASS thickness estimation algorithm.\\

\section{Recommendations for Future Work}
\subsection{The future of the MASS}

It is the authors opinion that the future of the MASS can follow two diverging paths, one academic and one industrial. In its current form, the MASS is an excellent research tool for implementing and testing algorithms for use in mobile robotics. Its hardware is capable of generating IMU position estimates, LiDAR scans, and point cloud scans as well as interacting with the environment through the use of its 6-DOF manipulator. The software can autonomously navigate, generate maps, localize, scan surfaces, simulate shotcrete application, and measure distances between point clouds. It is configured to run ROS and be controlled remotely, making it a suitable research tool for developing and testing novel mobile robotic algorithms. The MASS is not limited to mining research and can be used to develop novel solutions for many other applications, or simply used as a research tool to develop algorithms purely for the sake of research.\\

To follow and industrial path, the algorithms can be implemented on a new robot. The MASS is suitable for further testing and development that can take place in a real mine, but design changes must be made before it is ready to be deployed for regular use. Firstly, the MASS is built at one third scale so a larger robot is required for the final project. The skid steered wheels are suitable for the mine environment, but a tracked or articulated vehicle similar to a Load-Haul-Dump machine would likely perform better. The manipulator would also have to be replaced, either with a hydraulically actuated mechanism or a much larger and stronger version that has a higher ingress protection rating. The LiDAR and its nodding head mechanism would also have to be replaced with hardware more tolerant to the mine environment. The software was built in a modular way such that applying the algorithms on an entirely new robot would be as simple as possible. Alternate hardware drivers would be required, but the MASS software package will function as intended regardless of the hardware it uses.\\

\subsection{Localization}

The localization algorithm requires modification to the environment, but the mine environment if frequently modified for other purposes. Existing markers like rock bolts, survey markers, or spray paint can be used instead of placing markers manually. When testing begins in an actual mine, the feasibility and accuracy of using alternate keypoints to form fiducial markers can be investigated. If the mine has enough non-uniformity that surface features can be identified, no modification is necessary to generate keypoints for the fiducial markers. The localization algorithm uses the locations of the keypoints to determine the markers' locations. The keypoints themselves can be anything as long as they can be reliably detected. Alternate keypoint detection algorithms can be used without the need to modify the localization algorithm.\\

For this work, a fiducial marker consists of three keypoints, however that is simply the minimum number of keypoints necessary. The algorithm can easily be modified to use more keypoints per marker, but since multiple markers can be used it is not necessary as the additional keypoints can be used to create additional markers. An early approach to localization used established registration techniques to align point clouds that have been filtered for intensity. Using the registration approach unique shapes can be spray painted (similar to the one shown in Figure \ref{fig:minepaint}) and used as markers instead of three keypoints. Initial testing of the registration approach yielded poor results, so the keypoint based markers were used instead. Since localization was not the sole purpose of this research, a simpler yet more accurate method was chosen. If the mine operators truly desire to use existing markings for localization, the registration approach should be revisited.\\

\subsection{Trajectory Generation}

For the purposes of this application it is entirely reasonable to assume the operator would know the approximate height of the mine drift. To reduce the algorithms complexity the operator will enter this information in to the configuration file of the trajectory generation algorithm. This parameter can be obtained using an algorithm that examines the surface curvature and normal to determine whether to treat the surface section as a wall or a roof. This feature was left out of the current design since the drift dimensions will be readily available and autonomously determining it would reduce the reliability of the algorithm. With further research, novel techniques in determining where to sample the surface for generating via points can be developed. If alternate applications for the trajectory generation algorithm are desired, there is much research that can be done to address the specific requirements of that application. Though not required for this research, it is possible to develop the trajectory generation algorithm further to allow it to perform in any environment on any surface with any curvature. To do so, one must develop algorithms for segmenting the surface based on its orientation relative to the robot, and choosing the appropriate portion of the trajectory generation algorithm to apply.\\

\subsection{Thickness Estimation}

Novel thickness estimation techniques would be an entire research project in and of its own. The point cloud comparison algorithm used was chosen from a well developed field of study, with many alternatives available for implementation if the use case changes. If the user intends to perform additional processing of the point clouds, the CloudCompare software is an open-source project capable of executing many of the point cloud analysis algorithms presented in research. Alternatively, the Point Cloud Library (PCL) can be used to develop custom applications. The MASS already uses the Point Cloud Library so implementing functions from PCL is simple and straightforward.\\