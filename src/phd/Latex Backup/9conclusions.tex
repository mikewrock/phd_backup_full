\chapter{Conclusions and Recommendations for Future Work}
\label{chap:conclusions}
\section{Conclusions}
\enlargethispage{\baselineskip}
The mining industry is slow to adopt new technologies, but the testing results provided by the MASS shows the algorithms used hold great promise in automating aspects of the mining process. Not only can the MASS achieve greater consistency in the tasks it automates, but offers much better verification for the work it does. As well, it offers the benefit of removing the operator from a hazardous environment, allowing them to work remotely. Between 2012 and 2017, there were 17 deaths and 169 critical injuries in the Ontario mining industry as reported by the Ministry of Labour \cite{seebelow}. If this work could save a single life, the economic benefits would pale in comparison. Though worker safety is of utmost importance, a secondary goal of this work was to reduce the financial cost of mining, through efficiency, reduced training time, and/or increased operational availability. A conclusive answer to whether these goals have been achieved cannot be given until the system has been deployed in a mine, but the research indicates that both worker safety and mine efficiency will increase with the use of the MASS or its algorithms.\\
%https://www.ontario.ca/document/mining-sector-plan-2017-2018/mining-enforcement-statistics

The fiducial marker localization algorithm was developed for use in underground mining where \acrshort{gps} is not available. Localization can be achieved without the use of fiducial markers, but installing them in the mine yields far greater accuracy than alternative methods. The fiducial marker keypoints were chosen for their highly reflective properties, allowing them to be reliably and consistently detected.\\

The localization algorithm can be used by much more than just the MASS. Any autonomous vehicle in the mine can use the fiducial markers for localization. The markers can be used for loop closure, a common problem in \acrshort{slam} algorithms. When a robot navigates using \acrshort{slam} and drives in a loop, the algorithm often has errors in the map when it completes the loop. Upon completion of the loop, the algorithm must reconcile the difference between the environment it currently sees versus the map it previously built in a process called loop closure. Since \acrshort{slam} generates position estimates based on previous information, errors can build over time but do not grow unbounded since the landmarks used are absolute references. Having a highly accurate absolute reference point allows for much more accurate loop closure.\\

The whole purpose of mining is to extract material from the earth. To do so, a large volume of material must be transported and often requires processing. Transporting material from the removal site to the processing site is another task well suited for automation. To automate the transport of material, localization is necessary. Vehicles that autonomously transport material, equipment, or workers can benefit from the use of this localization algorithm.\\

The marker keypoints can be used for more than localization as well. Since every scan is localized to a common coordinate frame, the point clouds can simply be combined to produce a 3D model of the mine. This information is useful for visualization, planning, backfilling, and monitoring of the mine. Over time the weight of the earth compresses the mine in a process called rockmass convergence. By comparing initial parameters of the marker with scans taken at a later date, the relative motion of the keypoints can be used to measure uneven rockmass convergence. If the keypoints are fixed such that they do not move as the rockmass converges, they can be used to generate a second 3D model of the mine at a later date. Using the thickness estimation algorithm and the two 3D models of the mine the amount of change can be displayed. Alternatively, keypoints can be placed in areas where rockmass convergence is not likely to occur, such as the floor or the entrance to the mine. The system can determine how the keypoints have moved relative to the fixed keypoint locations.\\

Trajectory generation for shotcrete application and radiation scanning yielded consistent and satisfactory results. Some of the initial concepts seemed more elegant and robust, but in practice the simpler approach was more reliable and effective. The parameters of the spray pattern are easily changed, allowing users to apply the algorithm to many other applications. Alternate sensors or mobile-manipulator systems can be used with the algorithms to perform tasks like temperature mapping of a surface, acoustic sampling, or measuring the penetration of electromagnetic signals through a surface. The shotcrete and scanning tool changer developed at the \acrshort{mars} Laboratory can be replaced or modified allowing the trajectory generation algorithm to be applied to tasks like painting, coating, or washing a surface.\\

The thickness estimation algorithm implemented in \acrshort{ros} allows users a seamless experience when applying shotcrete and measuring its thickness. The framework built in this research allows for the possibility of implementing existing or novel point cloud comparison algorithms within the \acrshort{ros} environment. Though alternate point cloud comparison software is available, as of the time of publication none have been implemented in \acrshort{ros} other than those discussed herein. With the ability to compare previously saved point clouds or selected portions of them, the point cloud comparison algorithm can be used as-is for applications other than shotcrete thickness estimation, such as rockmass convergence, erosion calculations, or forest canopy growth. As long as two point clouds can be generated, they can be compared using the MASS thickness estimation algorithm.\\

When the MASS is deployed in an underground uranium mine, the robot or its algorithms will perform a number of tasks. Initially, as the robot enters the mine it will begin performing \acrshort{slam} to build a map as it travels through the mine to the work area. As the robot builds the \acrshort{slam} map, it can periodically perform a \acrshort{lidar} scan to build a 3D map of the mine as well as the 2D map generated through \acrshort{slam}. For greater accuracy, localization of other robots, and drift shape monitoring fiducial markers can be installed. Once the robot reaches the drift face it can make use of its tool changer's shielded radiation detector and the trajectory generation algorithm to scan the surface. Recording the location of the sensor and the measured radiation intensity, the information is later analyzed by geologists for use in mine planning. After installing and recording the fiducial markers the MASS is ready to apply shotcrete to the drift. The operator will select an area to apply shotcrete to using the \acrshort{gui} or initiates autonomous shotcrete mode if the area is larger than can be shown in a single scan. The MASS will perform an initial point cloud scan to determine the contours of the mine before applying shotcrete. Using the trajectory generation algorithm the shotcrete is applied in a consistent and reliable manner. Once the robot completes the shotcreting task, navigating along the wall as required, it performs another point cloud scan of the mine. The thickness estimation algorithm calculates and displays the achieved shotcrete thickness and, if necessary, performs additional shotcreting to guarantee a minimum shotcrete thickness. The \acrshort{slam} map, 3D mine model, and fiducial markers can be used by other robotic systems for localization and navigation as automation in the mine increases.\\

The MASS or a similar robot using the MASS software cannot function without the algorithms developed in this work. Localization is crucial for a robot to navigate the mine, but is even more important when performing shotcrete thickness estimation using localization for point cloud registration. Using the localization and navigation systems developed for the MASS it is capable of approaching the drift face and positioning or re-positioning itself in locations necessary to perform its instructed task. The localization system is also integral in the process of building a 3D model of the mine.\\

The two tasks the MASS is able to perform while removing workers from the hazardous environment are accomplished using the trajectory generation algorithm. Radiation scanning and shotcrete application is typically done by one worker at a time and, though replacing the worker with a robot removes them from the hazardous environment, the system still requires a human in the loop. It is not the intention to replace workers with robots, but simply to relocate the worker from a hazardous environment to a safer one. The skill and experience of a shotcrete worker is no longer necessary since the MASS performs reliably and consistently, and perhaps more importantly the work performed by the MASS can be easily verified unlike work performed by a human operator. Training time to operate the MASS is significantly less than the time it requires to master shotcrete application. The ability to operate the MASS remotely allows the operators to work from not only a safe but a convenient location and the task of operating the robot can be performed by multiple workers on rotation while the MASS operates continuously.\\

The most important impact the MASS will have on the mining industry will be with respect to worker safety. The life and health of the workers that mine the ore used in all aspects of the modern world are far more important than any robotic system that can be built. Mines may collapse, robots may be crushed, or environmental hazards may render the robot inoperable but the algorithms developed for these robots will further be developed as long as they continue to improve mine safety and working conditions in an economical way. The ability to perform more accurately and reliably makes using the MASS or its algorithms not only ethical but economical as well. Canadian engineers are constantly reminded of their obligation to society by wearing an ``Iron Ring'', an obligation that was not overlooked in the development of the MASS. Whether the algorithms developed for the MASS are used, or the MASS is used to develop and test new algorithms, it offers the potential to benefit society by reducing the risk workers must take to extract valuable materials from the earth.\\

\section{Recommendations for Future Work}
\subsection{The Future of the MASS}

It is the author's opinion that the future of the MASS can follow two diverging paths, one academic and one industrial. In its current form, the MASS is an excellent research tool for implementing and testing algorithms for use in mobile robotics. Its hardware is capable of generating \acrshort{imu} position estimates, \acrshort{lidar} scans, and point cloud scans as well as interacting with the environment through the use of its 6-\acrshort{dof} manipulator. The software can autonomously navigate, generate maps, localize, scan surfaces, simulate shotcrete application, and measure distances between point clouds. It is configured to run \acrshort{ros} and can be controlled remotely, making it a suitable research tool for developing and testing novel mobile robotic algorithms. The MASS is not limited to mining research and can be used to develop novel solutions for many other applications or simply used as a research tool to develop algorithms purely for the sake of research.\\

To follow an industrial path, the algorithms can be implemented on a new robot. The MASS is suitable for further testing and development that can take place in a real mine, but design changes must be made before it is ready to be deployed for regular use. Firstly, the MASS is built at one third scale so a larger robot is required for an industrial setting. The skid steered wheels are suitable for the mine environment, but a tracked or articulated vehicle similar to a Load-Haul-Dump machine would likely perform better. The manipulator would also have to be replaced, either with a hydraulically actuated mechanism or a much larger and stronger version that has a higher ingress protection rating. The \acrshort{lidar} and its nodding head mechanism would also have to be replaced with hardware more tolerant to the mine environment. The software was built in a modular way such that applying the algorithms on an entirely new robot would be simple. Alternate hardware drivers would be required, but the MASS software package will function as intended regardless of the hardware it uses.\\

Figure \ref{fig:masslay} illustrates the MASS software package. To implement the MASS algorithms on a different vehicle, the Husky UGV package must be replaced with a suitable package for the alternate vehicle chosen. If the system's hardware is not the same as the MASS, the peripheral and corresponding nodes will need to be modified as well. Alternate point cloud measurement devices, such as time of flight cameras, may be used in place of \acrshort{lidar}, but in order to navigate and perform \acrshort{slam}, additional data processing of the point cloud or alternate \acrshort{slam} methods would be required. To modify an existing shotcrete machine to perform autonomous shotcrete application the system would require a method of obtaining point clouds of its surroundings. Since current shotcrete machines require the operator to manually aim the spray nozzle, there is no need for closed loop control of its pose. Autonomous operation would require the system to be able to position the shotcrete spray nozzle at a desired position and orientation, which requires closed loop control to avoid a significant loss of accuracy. Closed loop control can be achieved using any sensor capable of measuring the extension or rotation of the actuated components, such as linear or rotary encoders.\\

\begin{figure}[h]
    \centering
\includegraphics[width=\textwidth]{Pics/mass_layout.pdf} 
    \caption{Visual Layout of MASS and Accompanying Packages}
    \label{fig:masslay}
\end{figure}

\subsection{Localization}

The localization algorithm requires modification to the environment, but the mine environment is frequently modified for other purposes. Existing markers like rock bolts, survey markers, or spray paint can be used instead of placing markers manually. When testing begins in an actual mine, the feasibility and accuracy of using alternate keypoints to form fiducial markers can be investigated. If the mine has enough non-uniformity that surface features can be identified, no modification is necessary to generate keypoints for the fiducial markers. The localization algorithm uses the locations of the keypoints to determine the markers' locations. The keypoints themselves can be anything as long as they can be reliably detected. Alternate keypoint detection algorithms can be used without the need to modify the localization algorithm.\\

For this work, a fiducial marker consists of three keypoints, however, that is simply the minimum number of keypoints necessary. The algorithm can easily be modified to use more keypoints per marker, but since multiple markers can be used it is not necessary as the additional keypoints can be used to create additional markers. An early approach to localization used established registration techniques to align point clouds that have been filtered for intensity. Using the registration approach unique shapes can be spray painted and used as markers instead of three keypoints. Initial testing of the registration approach yielded poor results, so the keypoint based markers were used instead. Since localization was not the sole purpose of this research, a simpler yet more accurate method was chosen. If the mine operators truly desire to use existing markings for localization, the registration approach should be revisited.\\

\subsection{Trajectory Generation}

For the purposes of this application it is entirely reasonable to assume the operator would know the approximate height of the mine drift. To reduce the algorithms complexity the operator will enter this information in the configuration file of the trajectory generation algorithm. This parameter can be obtained using an algorithm that examines the surface curvature and normal to determine whether to treat the surface section as a wall or a roof. This feature was left out of the current design since the drift dimensions will be readily available and autonomously determining it would reduce the reliability of the algorithm. With further research, novel techniques in determining where to sample the surface for generating via points can be developed. If alternate applications for the trajectory generation algorithm are desired, there is much research that can be done to address the specific requirements of that application. Though not required for this research, it is possible to develop the trajectory generation algorithm further to allow it to perform in any environment on any surface with any curvature. To do so, one must develop algorithms for segmenting the surface based on its orientation relative to the robot and choosing the appropriate portion of the trajectory generation algorithm to apply.\\

\subsection{Thickness Estimation}

Novel thickness estimation techniques would be an entire research project well beyond the scope of this work. The point cloud comparison algorithm used was chosen from a well developed field of study, with many alternatives available for implementation if the use case changes. If the user intends to perform additional processing of the point clouds, the CloudCompare software is an open-source project capable of executing many of the point cloud analysis algorithms presented in this research. Alternatively, the Point Cloud Library (\acrshort{pcl}) can be used to develop custom applications. The MASS already uses the Point Cloud Library so implementing functions from \acrshort{pcl} is simple and straightforward.\\