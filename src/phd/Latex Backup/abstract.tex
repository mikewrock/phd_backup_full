\addcontentsline{toc}{chapter}{Abstract}
\abstracter
\begin{spacing}{1.5}
An approach to improving worker safety in underground uranium drift mining through automating the radiation scanning and shotcrete application process is presented. Utilizing the Robot Operating System (\acrshort{ros}) framework to achieve autonomous operation, workers can be removed from the most hazardous areas of the mine. Worker knowledge and experience is utilized when instructing the system from a remote location. The current approach to radiation scanning exposes the workers to radiation hazards and employs the use of cumbersome handheld radiation sensors that are difficult to position accurately. Sprayable concrete liner known as shotcrete is applied to all surfaces of the drift to protect workers from hazards, such as rock fall and radon gas. Proper application of shotcrete and visual estimation of its thickness require highly skilled operators. This research presents a trajectory generation algorithm that is capable of autonomously applying shotcrete to unknown surface geometries. The trajectory generation algorithm can also be used to perform radiation scanning with the use of a previously developed tool changer. A novel localization algorithm developed for use in underground mining uses the Point Cloud Library (\acrshort{pcl}) to accurately register point cloud scans of the drift from before and after shotcrete application and produces thickness estimates of the entire shotcrete area. Estimating shotcrete thickness allows for verification to ensure worker safety and eliminates the human error present in current thickness estimation methods. The hardware and software systems are highly modular in order to allow components or algorithms to easily be replaced or implemented on other systems.  The localization system presented uses fiducial markers that allow the robot to generate 3D point cloud representations of the entire mine. While scanning the drift face, a shielded radiation sensor can be used for more directional measurements, and the locations at which the measurements are taken relative to the drift face can be recorded with great accuracy. Verifying shotcrete thickness, mapping mine drifts, and protecting workers from hazardous environments are three factors which contribute to improved safety and monitoring of underground uranium mines.
\end{spacing}